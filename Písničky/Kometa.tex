\hlavicka{Kometa}{J. Nohavica}

3/4 takt

\sloka{
\Ami Spatřil jsem kometu, \X oblohou letěla,\\
\X chtěl jsem jí zazpívat, \X ona mi zmizela,\\
\Dmi zmizela jako laň \GSedm u lesa v remízku,\\
\C v očích mi zbylo jen \ESedm pár žlutých penízků.
}

\sloka{
Penízky ukryl jsem do hlíny pod dubem,\\
až příště přiletí, my už tu nebudem,\\
my už tu nebudem, ach, pýcho marnivá,\\
spatřil jsem kometu, chtěl jsem jí zazpívat.
}

\refren{
\Ami O vodě, \X o trávě, \Dmi o lese,\quad\X\\
\GSedm o smrti, \X se kterou smířit \C nejde se,\quad\X\\
\Ami o lásce, \X o zradě, \Dmi o světě\quad\X\\
\ESedm a o všech lidech, co \X kdy žili na téhle \Ami planetě.\quad\X
}

\sloka{
Na hvězdném nádraží cinkají vagóny,\\
pan Kepler rozepsal nebeské zákony,\\
hledal, až nalezl v hvězdářských triedrech\\
tajemství, která teď neseme na bedrech.
}

\sloka{
Velká a odvěká tajemství přírody,\\
že jenom z člověka člověk se narodí,\\
že kořen s větvemi ve strom se spojuje\\
a krev našich nadějí vesmírem putuje.
}

\refren{Na na na \dots}

\sloka{
Spatřil jsem kometu, byla jak reliéf\\
zpod rukou umělce, který už nežije,\\
šplhal jsem do nebe, chtěl jsem ji osahat,\\
marnost mne vysvlékla celého donaha.
}

\sloka{
Jak socha Davida z bílého mramoru\\
stál jsem a hleděl jsem, hleděl jsem nahoru,\\
až příště přiletí, ach, pýcho marnivá,\\
já už tu nebudu, ale jiný jí zazpívá.
}

\refren{
O vodě, o trávě, o lese,\\
o smrti, se kterou smířit nejde se,\\
o lásce, o zradě, o světě,\\
bude to písnička o nás a kometě\dots
}
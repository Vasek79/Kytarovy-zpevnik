\hlavicka{Morituri te salutant}{Karel Kryl}

\sloka{
Cesta je \Ami prach a \G štěrk a \Dmi udusaná\\
\Ami hlína \C a šedé \F šmouhy \GSedm kreslí do vla\C sů\\
a z hvězdných \Dmi drah má \G šperk co \C kamením\\
se \Emi spíná \Ami a pírka \G touhy\\
z \Emi křídel pega\Ami sů.
}

\sloka{
Cesta je bič, je zlá jak pouliční dáma, má\\
v ruce štítky a pase staniol, a z očí chtíč jí plá,\\
když háže do neznáma, dvě křehké snítky\\
rudých gladiol.
}

\refren{
\G Seržante písek je bílý jak paže Daniely\\
\Ami počkejte chvíli mé oči uviděli\\
\G tu strašně dávnou vteřinu zapomnění\\
\Ami Seržante mávnou \GSedm a budem zasvěceni.\\
\C Morituri te salutant, \E morituri te salutant\dots
}

\sloka{
Tou cestou dál jsem šel, kde na zemi se zmítá\\
a písek víří křídla holubí a marš mi hrál\\
zvuk děl co uklidnění skýtá,\\
a zvedá chmýří které zahubí.
}

\sloka{
Cesta je fér a prach a udusaná hlína\\
mosazná včelka od vlkodlaka\\
Rezavý kvér, můj prach a sto let stará špína\\
a děsně velká bíla oblaka\dots
}
\hlavicka{Jaro}{Fešáci}

\sloka{
\Ami My čekali \C jaro, a \G zatím přišel \Ami mráz,\quad\X\\
tak strašlivou zimu nepoznal nikdo z nás,\\
z těžkých černých mraků se stále sypal sníh\\
a vánice sílí v poryvech ledových.\\
\C Z chýší dřevo \X mizí a \G mouka ubý\X vá,\\
\Dmi do sýpek se \X raději už \G nikdo nedí\X vá,\\
\C zvěř z okolních \X lesů nám \G stála u dve\X ří\\
\Dmi a hladoví \X ptáci při\G létli za zvě\X ří, a stále \Ami blíž.\quad\X
}

\sloka{
Tak jednoho dne večer, to už jsem skoro spal,\\
když vystrašenej soused na okno zaklepal:\\
\uv{Můj chlapec doma leží, v horečkách vyvádí,\\
já do města bych zajel, doktor snad poradí.}\\
Půjčil jsem mu koně, a když sedlo zapínal,\\
dříve, než se rozjel, jsem ho ještě varoval:\\
\uv{Nejezdi naší zkratkou, je tam velkej sráz\\
a v týhletý bouři tam snadno zlámeš vaz, tak neriskuj!}
}

\sloka{
Na to smutné ráno dnes nerad vzpomínám,\\
na tu hroznou chvíli, když kůň se vrátil sám,\\
trvalo to dlouho, než se vítr utišil,\\
na sněhové pláně si každý pospíšil.\\
Jeli jsme tou zkratkou až k místu, které znám,\\
kterým bych v té noci nejel ani sám,\\
a pak ho někdo spatřil, jak leží pod srázem,\\
krev nám tuhla v žilách nad tím obrazem, já klobouk sňal.
}

\sloka{
\Ami Někdy ten, kdo \C spěchá, se \G domů nevra\Ami cí\dots
}
\zvyraznit
\hlavicka[Richard Müller]{Srdce jako kníže Rohan}

\sloka{
\F Měsíc je jak Zlatá bula \C sicilská\\
\Ami stvrzuje, že kdo chce, ten se \G dopíská.\\
\F Pod lampou jen krátce, v přítmí \C dlouze zas,\\
\Ami otevře ti Kobera a \G můžeš mezi \F nás.\CSiroky\\
}

\slokaBezCisla{
\AmiSiroky\GSiroky\FSiroky\CSiroky\AmiSiroky\GSiroky
}

\sloka{
Moje teta, tvoje teta, parole,\\
dvaatřicet karet křepčí na stole.\\
Měsíc svítí sám a chleba nežere,\\
ty to ale koukej trefit, frajere. Protože\\
}

\refrenX{1}{
\F dnes je valcha u starýho \C Růžičky,\\
\Ami dej si prachy do pořádný \G ruličky.\\
\F Co je na tom, že to není \C extra nóbl byt?\\
\Ami Srdce jako kníže Rohan \G musíš mít.\\
}

\slokaBezCisla{
\FSiroky\CSiroky\AmiSiroky\GSiroky\akord{$2\times$}
}

\sloka{
Ať jsi přes den docent nebo tunelář,\\
herold svatý pravdy nebo jinej lhář,\\
tady na to každej kašle zvysoka,\\
pravda je jen jedna. Slova proroka říkaj, že
}

\refrenX{2}{
když je valcha u starýho Růžičky,\\
budou vcelku na nic všechny řečičky.\\
Buďto trefa, nebo kufr smůla, nebo šnit,\\
jen to srdce jako Rohan musíš mít.
}

\sloka{
Kdo se bojí, má jen hnědý kaliko,\\
možná občas nebudeš mít na mlíko.\\
Jistě ale poznáš co jsi vlastně zač,\\
svět nepatřil nikomu kdo nebyl hráč. A proto
}

\refrenX{3}{
Ať je valcha u starýho Růžičky,\\
nebo pouť až k tváři Boží rodičky.\\
Ať je válka, červen, mlha bouřka nebo klid,\\
srdce jako kníže Rohan musíš mít.
}

\refrenX{4}{
Dnes je valcha u starýho Růžičky,\\
když jsi malej, tak si stoupni na špičky,\\
malej nebo nachlapenej Cikán, Brňák, Žid -\\
srdce jako kníže Rohan musíš mít.\\
}

\slokaBezCisla{
\FSiroky\CSiroky\AmiSiroky\GSiroky\akord{$2\times$}
}

\refrenX{1}{
Dnes je valcha u starýho Růžičky, (to víš, že jo)\\
\dots
}
\refrenX{1}{}
\refrenX{3}{}

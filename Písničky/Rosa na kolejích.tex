\hlavicka{Rosa na kolejích}{Wabi Daněk}

\sloka{
T\C ak, jako jazyk st\FSest ále nar\FisSest áží\hspace{0.5em}\GSest na vylomený z\C ub,\\
tak se vracím k sv\FSest ýmu nádr\FisSest aží,\hspace{0.5em}\GSest abych šel zas d\C ál,\\
přede mnou st\FSest íny se dl\GSest ouží a n\Ami ad krajinou kr\Cdim ouží\\
podivnej pt\FSest ák, \FisSest ptá \GSest k nebo mr\C ak.
}

\refren{
Tak do toho šl\FSest ápni, ať v\GSest idíš kous\FSest ek sv\C ěta,\\
vzít do dlaně d\FSest álku z\GSest ase jedn\FSest ou zk\C us,\\
telegrafní dr\FSest áty hr\GSest ajou ti \FSest už l\C éta\\
to nekonečně dl\FSest ouh\FisSest ý mo\GSest not\FisSest ónní\FSest  bl \C ues,\\
je ráno, je ráno, nohama st\FSest írá\FisSest š\hspace{0.5em}ro\GSest su na k\FSest ole\FisSest jích.\C
}

\sloka{
Pajda dobře hlídá pocestný, co se nocí toulaj,\\
co si radši počkaj, až se stmí, a pak šlapou dál,\\
po kolejích táhnou bosí a na špagátě nosí\\
celej svůj dům, deku a rum.
}

\refren{
\dots\\
n\C ohama st\FSest írá\FisSest š\hspace{0.5em}ro\GSest su na k\FSest ole\FisSest jích\dots\C
}

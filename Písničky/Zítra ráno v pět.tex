\hlavicka[Jaromír Nohavica]{Zítra ráno v pět}

\sloka{
Až mě \Emi zítra ráno v pět, \G ke zdi postaví,\\
je\IC ště si napo\D sled dám\IG\ vodku na zdra\E ví,\\
z očí\IAmi\ pásku strhnu \D si, to abych\IG\ viděl na ne\Emi be\\
a\IAmi\ pak vzpomenu \HSedm si, \IEmi lásko, na tebe,\AmiSiroky\DSiroky\GSiroky\EmiSiroky\\
a \Ami pak vzpomenu \HSedm si na te\Emi be\dots\\
}

\sloka{
Až zítra ráno v pět přijde ke mně kněz,\\
řeknu mu, že se splet, že mně se nechce do nebes,\\
že žil jsem, jak jsem žil, a stejně tak i dožiju\\
a co jsem si nadrobil, to si i vypiju,\\
a co jsem si nadrobil, si i vypiju.
}

\sloka{
Až zítra ráno v pět poručík řekne: \uv{Pal!},\\
škoda bude těch let, kdy jsem tě nelíbal,\\
ještě slunci zamávám, a potom líto přijde mi,\\
že tě, lásko, nechávám, samotnou tady na zemi,\\
že tě, lásko, nechávám, na zemi.
}

\sloka{
Až zítra ráno v pět prádlo půjdeš prát\\
a seno obracet, já u zdi budu stát,\\
tak přilož na oheň a smutek v sobě skryj,\\
prosím, nezapomeň, nezapomeň a žij,\\
Lásko, nezapomeň a žij\dots
}

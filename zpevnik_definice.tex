% Nastaveni zrcadla tisku a ostatnich delkovych registru
\setlength{\textwidth}{170mm}
\setlength{\hoffset}{-1in}
\setlength{\voffset}{-0,54mm}
\setlength{\oddsidemargin}{15mm}
\setlength{\evensidemargin}{25mm}
\setlength{\parindent}{0mm}
\setlength{\headsep}{0mm}
\setlength{\headheight}{0mm}
\setlength{\topmargin}{0mm}
\setlength{\textheight}{242mm}
\setlength{\columnsep}{0mm}
\setlength{\footskip}{15mm}



% Predefinovani vypisu cisla stranky
%\makeatletter
%\def\ps@plain{\let\@mkboth\@gobbletwo                                           
%     \let\@oddhead\@empty\def\@oddfoot{\reset@font
%        \colorbox{podklad}{
%        \makebox[\textwidth][l]{\hfill {\bf \thepage} \hfill}}
%          \hfil}\let\@evenhead\@empty\let\@evenfoot\@oddfoot}
%\makeatother


% Predefinovani vypisu nadpisu nazvu pisnicek
\makeatletter
\renewcommand\section{\@startsection {section}{1}{\z@}%
                                     {-3.5ex \@plus -1ex \@minus -.2ex}%
                                     {.1ex \@plus.1ex}%
				      {}}
\makeatother


\newcounter{cislosloky}


% Definice,jak se má vysázet název písně a autor
\newcommand{\hlavicka}[2]{{
%\emph{\huge{#1}}\hfill \tiny{#2}\normalsize 
\section*{\raisebox{-2ex}{\emph{ \huge{#1}}}} \hfill \tiny #2
\normalsize
\vskip 3pt
\hrule height 3pt
\vskip 10pt
\setcounter{cislosloky}{0}

\addcontentsline{toc}{subsection}{#1 \texttt{\small (#2)}}

}}


% Definice prikazu "nadpisobsah" (pouziva se pouze pro nadpis "Obsah")
\newcommand{\nadpisobsah}[1]{{
 \fontsize{10}{10}
 \usefont{T1}{cmr}{m}{n}
 \selectfont

 %\colorbox{podklad}{\makebox[\textwidth]{\scalebox{3.5}{\color{podklad}X}}}

 \vspace{-10mm}
 \hfill \Large ~
 \vspace{-18mm}

 \section*{#1}

}}



% Definice prikazu "nadpisbezautora" - pouziva se pro nazvy kapitol bez druheho parametru (autora)
\newcommand{\nadpisbezautora}[1]{
{ \fontsize{10}{10}
 \usefont{T1}{ptl}{m}{n}
 \selectfont

 \setcounter{cislosloky}{0}

 \colorbox{podklad}{\makebox[\textwidth]{\scalebox{3.5}{\color{podklad}X}}}

 \vspace{-10mm}
 \hfill \Large ~
 \vspace{-18mm}

 \section*{#1}
 \addcontentsline{toc}{subsection}{#1}

}}

% Zpusob zapisu nadpisu "Obsah" do seznamu vsech kapitol v obsahu
\makeatletter
\renewcommand\tableofcontents{%
     \nadpisobsah{\contentsname}
     \@starttoc{toc}%
}

\makeatother


\interlinepenalty=10000


\newcommand{\sloka}[1]{
\setlength{\leftmargin}{20mm}
\begin{list}{\refstepcounter{cislosloky}\thecislosloky.}{\setlength{\leftmargin}{10mm}}                                                                                    
  \fontencoding{T1}\selectfont
  \item #1
\end{list}
}


\newcommand{\slokanocis}[1]{
%\setlength{\leftmargin}{20mm}
\begin{list}{}{\setlength{\leftmargin}{10mm}}
  \fontencoding{t}\selectfont
  \item #1
\end{list}
}


\newcommand{\refren}[1]{
\begin{list}{Ref:}{\setlength{\leftmargin}{10mm} \setlength{\labelwidth}{10mm}}
  \fontencoding{T1}\selectfont
  \item #1
\end{list}
}

\newcommand{\refrenI}[1]{
\begin{list}{Ref 1:}{\setlength{\leftmargin}{10mm} \setlength{\labelwidth}{10mm}}
  \fontencoding{T1}\selectfont
  \item #1
\end{list}
}

\newcommand{\refrenII}[1]{
\begin{list}{Ref 2:}{\setlength{\leftmargin}{10mm} \setlength{\labelwidth}{10mm}}
  \fontencoding{T1}\selectfont
  \item #1
\end{list}
}

\newcommand{\repetice}[1]{
\begin{list}{\bf [:}{\setlength{\leftmargin}{5mm} \setlength{\topsep}{-0.3em}}
  \fontencoding{T1}\selectfont
  \item #1{\bf :]}
\end{list}
}


\newcommand{\repeticenoindent}[1]{
\begin{list}{\bf [:}{\setlength{\leftmargin}{5mm} \setlength{\topsep}{-0.3em}}
  \fontencoding{T1}\selectfont
  \item #1{\bf :]}
\end{list}
}


\newcommand{\outro}[1]{
\begin{list}{Outro:}{\setlength{\leftmargin}{10mm} \setlength{\labelwidth}{20mm}}
  \fontencoding{T1}\selectfont
  \item #1
\end{list}
}


%\newcommand{\repeticenoindent}[1]{
%{\bf [:~}#1{\bf ~:]}
%}


\newcommand{\carauprostred}{%
  \vspace{-3\baselineskip}

  \begin{center}
  \rule[0mm]{30mm}{0.1mm}
  \end{center}
}

% Predefinovani vypisu nadpisu nazvu pisnicek
\makeatletter
\renewcommand\section{\@startsection {section}{1}{\z@}%
                                     {-3.5ex \@plus -1ex \@minus -.2ex}%
                                     {.1ex \@plus.1ex}%
				                     {}}
\makeatother


\newcounter{cislosloky}


% Definice,jak se má vysázet název písně a autor
\newcommand{\hlavicka}[2]{{
\section*{\raisebox{-2ex}{\emph{ \huge{#1}}}} \hfill #2
\normalsize
\vskip 3pt
\hrule height 3pt
\vskip 10pt
\setcounter{cislosloky}{0}
\addcontentsline{toc}{subsection}{#1 \texttt{\small (#2)}}
}}


% Definice prikazu "nadpisobsah" (pouziva se pouze pro nadpis "Obsah")
\newcommand{\nadpisobsah}[1]{{
 \fontsize{10}{10}
 \usefont{T1}{cmr}{m}{n}
 \selectfont
 \vspace{-10mm}
 \hfill \Large ~
 \vspace{-18mm}
 \section*{#1}
}}


% Definice prikazu "nadpisbezautora" - pouziva se pro nazvy kapitol bez druheho parametru (autora)
\newcommand{\nadpisbezautora}[1]{
{ \fontsize{10}{10}
 \usefont{T1}{ptl}{m}{n}
 \selectfont
 \setcounter{cislosloky}{0}
 \colorbox{podklad}{\makebox[\textwidth]{\scalebox{3.5}{\color{podklad}X}}}
 \vspace{-10mm}
 \hfill \Large ~
 \vspace{-18mm}
 \section*{#1}
 \addcontentsline{toc}{subsection}{#1}
}}


% Zpusob zapisu nadpisu "Obsah" do seznamu vsech kapitol v obsahu
\makeatletter
\renewcommand\tableofcontents{%
     \nadpisobsah{\contentsname}
     \@starttoc{toc}%
}


\makeatother


\interlinepenalty=10000


\newcommand{\setTextFont}{\fontencoding{T1}\fontfamily{fav}\selectfont}


\newcommand{\sloka}[1]{
\setlength{\leftmargin}{20mm}
\begin{list}{\textbf{\emph{\refstepcounter{cislosloky}\thecislosloky.}}}{\setlength{\leftmargin}{10mm}}
  \setTextFont
  \item #1
\end{list}
}


\newcommand{\slokanocis}[1]{
\begin{list}{}{\setlength{\leftmargin}{10mm}}
  \setTextFont
  \item #1
\end{list}
}


\newcommand{\slokaopakovani}[1]{
\begin{list}{\textbf{\emph{#1}}}{\setlength{\leftmargin}{10mm}}
  \setTextFont
  \item
\end{list}
}


\newcommand{\refren}[1]{
\begin{list}{\textbf{\emph{R:}}}{\setlength{\leftmargin}{10mm} \setlength{\labelwidth}{10mm}}
  \setTextFont
  \item #1
\end{list}
}


\newcommand{\refrenI}[1]{
\begin{list}{\textbf{\emph{R1:}}}{\setlength{\leftmargin}{10mm} \setlength{\labelwidth}{10mm}}
  \setTextFont
  \item #1
\end{list}
}


\newcommand{\refrenII}[1]{
\begin{list}{\textbf{\emph{R2:}}}{\setlength{\leftmargin}{10mm} \setlength{\labelwidth}{10mm}}
  \setTextFont
  \item #1
\end{list}
}


\newcommand{\refrenIII}[1]{
\begin{list}{\textbf{\emph{R3:}}}{\setlength{\leftmargin}{10mm} \setlength{\labelwidth}{10mm}}
  \setTextFont
  \item #1
\end{list}
}


\newcommand{\repetice}[1]{
\begin{list}{\bf [:}{\setlength{\leftmargin}{5mm} \setlength{\topsep}{-0.3em}}
  \setTextFont
  \item #1{\bf :]}
\end{list}
}


\newcommand{\repeticenoindent}[1]{
\begin{list}{\bf [:}{\setlength{\leftmargin}{5mm} \setlength{\topsep}{-0.3em}}
  \setTextFont
  \item #1{\bf :]}
\end{list}
}

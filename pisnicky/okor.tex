 
\hlavicka{Okoř}{}

\sloka{
 \C Na Okoř je cesta, jako žádná ze sta,\\
   \GSedm roubená je strom\C ma.\\
   Když jdu po ni v létě, samoten na světě,\\
   \GSedm sotva pletu noha\C ma.\CSedm \\
   \F Na konci té cesty \C trnité\\
   \G stojí krčma jako hrad.\\
   \C Tam se sešli trempi, hladoví a sešlí,\\
   \GSedm začli sobě noto\C vat.
   }
   
\refren{
   Na hradě Okoři, \GSedm světla už nehoří, \\
   \C bílá paní \GSedm šla už dávno \C spát.\\
   Ta měla ve zvyku \GSedm podle svého budíku\\
   \C o půlnoci \GSedm chodit straší\C vat \CSedm .\\
   \F Od těch dob, co jsou tam \C trampové,\\
   \DSedm musí z hradu \G pryč.\\
   \C A tak žije v podhradí, \GSedm se šerifem dovádí,\\
   \C on ji sebral \GSedm od komůrky \C klíč.
   }

\sloka{
   Jednoho dne z rána, roznesla se fáma,\\
   že byl Okoř vykraden. \\
   Nikdo neví do dnes, kdo to tenkrát odnes,\\
   nikdo nebyl dopaden.\\
   Šerif hrál celou noc mariáš\\
   s Bílou paní v kostnici.\\
   Místo aby hlídal, zuřivě ji líbal,\\
   dostal z toho zimnici.
   }

   \refren{
   Na hradě Okoři\ldots...}